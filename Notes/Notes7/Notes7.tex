\documentclass{article}

\usepackage{algorithmic, amsmath, amsthm, amsfonts, amssymb, dsfont, enumerate, tikz, tikz-cd, color, mathrsfs,inputenc,epigraph} %tikz is for drawing lattices %tikz-cd is for commutative diagrams
															%color is for making notes in red 
															%mathrsfs is for power set font
%\usepackage[mathscr]{eucal} %mathscr gives nice script fonts

\newtheoremstyle{problemstyle}  % <name> This is my problemstyle. use begin{problem}.
        {12pt}                                               % <space above>
        {}                                               % <space below>
        {}                               % <body font>
        {}                                                  % <indent amount}
        {\bfseries}                 % <theorem head font>
        {\normalfont\bfseries.}         % <punctuation after theorem head>
        {.5em}                                          % <space after theorem head>
        {}                                                  % <theorem head spec (can be left empty, meaning `normal')>


\DeclareSymbolFont{bbold}{U}{bbold}{m}{n}
\DeclareSymbolFontAlphabet{\mathbbold}{bbold}

\theoremstyle{problemstyle}

\newtheorem{problem}{Problem}
\newtheorem{theorem}{Theorem}
\newtheorem{example}{Example}
\newtheorem{proposition}{Proposition}
\newtheorem{article}{Article}
\newtheorem{lemma}{Lemma}
\newtheorem{corollary}{Corollary}
\newtheorem{exercises}{Exercises}
\newtheorem{exercise}{Exercise}
\newtheorem{definition}{Definition}

\setlength\parindent{0pt}
\newcommand{\Mod}[1]{\ (\mathrm{mod}\ #1)}
\newcommand{\ndiv}{\hspace{-4pt}\not|\hspace{2pt}}

\title{ \vspace{-10ex} %uncomment to remove vertical space
%title of assignment goes here e.g. "Math 721 Homework 3"
Math 313/623 Notes 7
}

\author{David L. Meretzky
}


\date{%date assignment is due goes here
Tuesday March 26th, 2019
} 

\renewcommand{\thefootnote}{\arabic{footnote}}
%\renewcommand{\thefootnote}{$\dagger$} %changes default footnote marking to a dagger instead of a number (numbers are sometimes mistaken for citations)

\begin{document}

\maketitle

\epigraph{We're gonna rock around the clock tonight}{Bill Haley}

In these notes we will investigate some important arithmetic functions and the M\"obius Inversion Formula. 

\section*{Back to Gauss Section 2}

\subsection*{Solutions of Congruences of the first degree}

\setcounter{article}{25}
\begin{article}
In the case $(a,m) = 1$ we know that $ax + b \equiv c \Mod{m}$. Suppose that $x_0$ solves this congruence, that is $ax_0 + b \equiv c \Mod{m}$. It should follow from article 9 that if for any $x_1 \in \mathbb{Z}$ such that  $x_1 \equiv x_0 \Mod{m}$, $x_1$ is also a solution to the congruence. Furthermore, if $t$ is any other root of the congruence $ax + b \equiv c \Mod{m}$, then it must be congruent to $x_0$. Spelling it all out, if $(a,m) = 1$, then given any solution $x_0$, all solutions are given by the set $x_0 + m\mathbb{Z}$.\\

I will now say the same thing in about 5 different ways. Pick whichever way makes sense to you. This justifies why we say that the equation $ax + b \equiv c \Mod{m}$ only has a single root. It's ``root" is the set of all numbers congruent to $x_0$ modulo $m$. More precisely, there is exactly one least residue of $x_0$ in the set $x_0 + m\mathbb{Z}$. More modernly, $x_0 + m\mathbb{Z}$ is a single element of $\mathbb{Z}/m\mathbb{Z}$, usually identitfied with the aforementioned least residue.\\

Note that these results do not hold if $(a,m) \neq 1$ or if the degree of the congruence is greater than $1$, say $ax^2 + b \equiv c \Mod{m}$. 
\end{article}

\begin{proof}
I wouldn't dare. 
\end{proof}

\begin{article}
Gauss makes a delightful but somewhat hard to parse observation that if we want to solve a congruence of the form $ax+t \equiv u \Mod{b}$, where now $(a,b) = 1$, we only need to solve $ax \equiv \pm 1 \Mod{b}$. We can easily check that that if there is some solution $r$ such that $ar \equiv \pm 1 \mod{b}$, then $\pm(u-t)r$ will be a solution to $ax+t \equiv u \Mod{b}$. That is, $$a\pm(u-t)r+t \equiv (u-t)+t \equiv u \Mod{b}$$ since $ar \equiv \pm 1 \mod{b}$. Furthermore, we note by the definition of congruence that solving  $ax \equiv \pm 1 \Mod{b}$ amounts to finding an $x,y \in mathbb{Z}$ such that $ax + by = \pm 1$.\\

Here we outline the algorithm for actually finding a solution to such a congruence again given the condition on $a$.  Note that the condition $(a,b) = 1$ is exactly enough to guarantee solutions $x,y \in \mathbb{Z}$. See the related discussion in Notes 2 definition 4: for fixed $(a,b) = 1$, the set of all possible linear combinations $\{ax+by|x,y \in \mathbb{Z}\} = (a,b) = (1) = \mathbb{Z}$ where the parentheses now denote ideals.\\ 

Start with $(a,b) = 1$ How do we find the associated $x$ and $y$ such that $ax+by = 1$? Apply Lemma $2$ of Notes $2$. Without loss of generality say that $a > b$. Then there exist unique $q_0$, $0< r_0 <b$ such that $a = bq_0 + r_0$. Note that $(a,b) = (b,r_0) = 1$, that is, $0< r_0 <b$ guarantees that $b$ and $r_0$ are relatively prime. Moreover, suppose there were $x_0$,$y_0$ such that $x_0r_0 + y_0b = 1$, then since $r_0 = a - bq_0$, we have  
\begin{align*}
x_0r_0 + y_0b =& x_0(a - bq_0) + y_0b \\
=& x_0a + (y_0 - q_0)b\\
=& 1
\end{align*}
which gives us our initial $x$ and $y$ as $x = x_0$, $y = (y_0 - q_0)$.\\

We are not finished however, this relies upon the assumption that we know how to solve $x_0r_0 + y_0b = 1$. Use the same process on this equation to obtain $x_0$ and $y_0$ in terms of some $x_1$ and $y_1$.  Again, begin by dividing $b$ by $r_0$.\\

This method must terminate eventually when some $r_n = 1$ at which point it becomes easy to find the $x_n$ and $y_n$ coefficents. This is called the Euclidean Algorithm. 
\end{article}

We skip article 28. 

\setcounter{article}{28}

\begin{article}
Now we investigate $ax + t \equiv u \Mod{m}$ the case where $a$, the coefficient on the unknown, is not relatively prime to the modulus $m$. Suppose $(a,m) = \delta > 1$. By article 5, if $x$ satisfies the congruence relative to the modulus $m$, then it certainly satisfies the congruence relative to the modulus $\delta$. However, $ax \equiv 0 \Mod{\delta}$ since $\delta|a$. Thus $t \equiv u \Mod{\delta}$ is the only case where the congruence has a solution. In this case, $\delta|t - u$. Thus applying article 22, we obtain $$(a/\delta)x  \equiv (u-t)/\delta \Mod{m/\delta}$$ since $a/\delta$ and $m/\delta$ are relatively prime we can use the machinery of the previous articles to solve this case now, i.e. the coefficient of $x$ and the modulus are relatively prime. 
\end{article}

In the rest of section $2$ Gauss proves the chinese remainder theorem and then uses this to give the simpler proof of Notes 6. Proposition 8 which is contained in the exercises of notes 6. 


\subsection*{Exercises}

\end{document}




