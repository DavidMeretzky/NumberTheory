\documentclass{article}
\usepackage{verbatim}
\setlength\parindent{0pt}

\begin{document}
	{\bfseries Hunter College
		\newline Spring 2019
		\newline Math 313/623: Theory of Numbers $\;\;\;\;$ 3 hrs, 3 cr. 
		\newline Tu Th 4:10pm-5:25pm, Hunter East 921
		\newline Instructor: David Meretzky
		\newline Email: dm594@hunter.cuny.edu
		\newline Office Hours: Th 3-4pm Dolciani Math Learning Center (7th floor library)
		}\\
	
	REQUIRED TEXTS: \textit{Disquisitiones Arithmeticae}, by Carl Gauss, translated by Arthur Clarke\\
     \textit{A Classical Introduction to Modern Number Theory}, by Kenneth Ireland and Michael Rosen\\
     \textit{Elementary Number Theory and its applications}, by Kenneth H. Rosen\\
	
	Course Discription: Number Theory is one of the oldest areas of Mathematics. It is also one of the most active research areas in modern Mathematics. This class will develop Number Theory simultaneously from classical and modern perspectives. The subject was intensively studied in the ancient world and in 1798 a young German Mathematician named Carl Fredrich Gauss consolidated, distilled, and extended the state of the art into the text we will be using, the \textit{Disquisitiones Arithmeticae}. Gauss's was so careful and brilliant that this text is still the state of the art as an introduction to the subject. The translation by Arthur Clarke is from the 80s and is supremely clear and readable. I will provide additional notes which supplement the text with examples and connect it to the modern text \textit{A Classical Introduction to Modern Number Theory}. \\
	
	We will for the most part be following the \textit{Disquisitiones Arithmeticae}. Each class I will assign reading for the next class. All students are expected to read both the text and my notes pretaining to that section of text. Classes where we discuss the \textit{Disquisitiones Arithmeticae} will be held in a round table discussion format. Two students will be randomly selected at the begining of class to mediate and guide the discussion. This will include reading aloud the articles from that week's portion of the text. Class participation and attendence will count heavily in your grade. 
	
	Periodically we will have classes in the usual lecture format where we discuss the modern texts.\\ 
	
	Prerequisites: Linear Algebra, Math 260. Some experience with Abstract Algebra (Math 311) will go a long way. If you have not taken Abstract Algebra it would be a good idea to read independently about groups and rings. I suggest Charles Pinter's classic and readable introduction \textit{A book of Abstract Algebra}. \\ 
	
	Homework: I will assign exercises in the notes and during class. Homework will not be collected but the problems will indicate what exam problems will be like. \\
	
	Exams: There will be two midterms and a final exam. The final will be cumulative. The final time will be announced later in the course. On an exam, one might be asked to state definitions, state and prove theorems, and solve problems similar to the homework.\\
	
	Attendance: Attendance is mandatory. If you do miss a class due to an emergency, you should get the notes and topics discussed from a classmate and \textbf{you must email me}. Attendance will be taken for record keeping. You will have three excused absences. Each additional absence will result in the deduction of a half letter grade from your final grade. \\
	
	Grading Policy: Participation is 30\%. This includes both participation when you are called as a mediator (20\%) and participation when you are not the mediator (10\%). There will be a rubric for your participation as a mediator. There will be two midterms and a final exam. The final exam will count as two exam grades. Thus there are 4 total exam grades. The lowest will be dropped. So the remaining 70\% is split evenly among the three exam grades.\\
	
\begin{center}	Tentative Class Schedule until the first exam: \end{center}

\centerline{
\begin{tabular}{ |c|c|c|c|c|c| } 
\hline
Class & Date & Format & Topic & Source & Section \\[0.5ex]
 \hline \hline
 Class 1 &  01/29 & Lecture & Syllabus and Congruences & Gauss and Notes 1 & Section 1 Article 1\\ 
 \hline
 Class 2 &  01/31 & Discussion & Congruences & Gauss and Notes 1 & Section 1 Articles 1-4 \\ 
 \hline
 Class 3 &  02/05 & Discussion & Congruences & Gauss and Notes 1 & Section 1 Articles 5-12 \\ 
 \hline
 Class 4 &  02/07 & Discussion & Prime Factors & Gauss and Notes 2 & Section 2 Articles 13-16 \\ 
 \hline
 College Closed & 02/12 & College Closed & College Closed & College Closed & College Closed \\ 
 \hline
 Class 5 &  02/14 & Lecture & Unique Factorization & Ireland and Rosen and Notes 2 & Chapter 1 Section 1 \\ 
 \hline
 Class 6 &  02/19 & Discussion & Prime Factors & Gauss and Notes 3 & Section 2 Articles 17-25 \\ 
 \hline
 Class 7 &  02/21 & Lecture & Rings & Notes 3 & Pages 4-7 \\ 
 \hline
 Class 7 &  02/21 & Lecture & Rings & Notes 3 & Pages 4-7 \\ 
 \hline
 Class 8 &  02/26 & Lecture & Unique Factorization in k[x] & Ireland and Rosen and Notes 4 & Chapter 1 Section 2 \\ 
 \hline
 Class 9 &  02/28 & Midterm 1 & \ \   & \ \ & \ \  \\ 
 \hline
\end{tabular}
}
	

\end{document}